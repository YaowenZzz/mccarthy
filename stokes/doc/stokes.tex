\documentclass[letterpaper,final,12pt,reqno]{amsart}

\usepackage[total={6.3in,9.2in},top=1.1in,left=1.1in]{geometry}

\usepackage{verbatim}
\usepackage{empheq}
\usepackage[dvipsnames]{xcolor}
\usepackage{animate}
\usepackage{graphicx}
\usepackage{fancyvrb}

%\usepackage{palatino}

% hyperref should be the last package we load
\usepackage[pdftex,
colorlinks=true,
plainpages=false, % only if colorlinks=true
linkcolor=blue,   % only if colorlinks=true
citecolor=Red,   % only if colorlinks=true
urlcolor=black     % only if colorlinks=true
]{hyperref}

\renewcommand{\baselinestretch}{1.05}

\newcommand{\ddt}[1]{\ensuremath{\frac{\partial #1}{\partial t}}}
\newcommand{\ddx}[1]{\ensuremath{\frac{\partial #1}{\partial x}}}
\newcommand{\ddy}[1]{\ensuremath{\frac{\partial #1}{\partial y}}}
\newcommand{\pp}[2]{\ensuremath{\frac{\partial #1}{\partial #2}}}
\renewcommand{\t}[1]{\texttt{#1}}
\newcommand{\Matlab}{\textsc{Matlab}\xspace}
\newcommand{\eps}{\epsilon}
\newcommand{\grad}{\nabla}
\newcommand{\Div}{\nabla\cdot}
\newcommand{\devstress}{\tau}
\newcommand{\RR}{\mathbb{R}}

\newcommand{\hbn}{\hat{\mathbf{n}}}

\newcommand{\bu}{\mathbf{u}}
\newcommand{\bv}{\mathbf{v}}

\newcommand{\bX}{\mathbf{X}}



\begin{document}
\graphicspath{{../figures/}}

\title{Appendix A: A finite element Stokes solver \\ for glacier flow}

\author{Ed Bueler}

\date{\today}

\maketitle

\renewcommand{\theequation}{A\arabic{equation}}


\section{Introduction}

These notes are an appendix to my notes \emph{Numerical modelling of glaciers, ice sheets, and ice shelves}---here called ``the notes''---which are used for the International Summer School in Glaciology in McCarthy, Alaska.

Here we restate the Stokes model for any Glen exponent $n$ and we derive the corresponding weak form.  Then we sketch how a finite element (FE) method \cite{Elmanetal2014} is set up using triangular elements for an arbitrary planar region.  We briefly discuss issues of stable mixed FE choices and of verification using slab-on-a-slope solutions.  Then we describe how Gmsh\footnote{\url{http://gmsh.info/}}, Firedrake\footnote{\url{https://www.firedrakeproject.org/}}, and PETSc\footnote{\url{http://www.mcs.anl.gov/petsc/}} are used in a numerical solution.


\section{Stokes equations and the weak form}

Recall the Glen-Stokes model in equations (3), (4), (5) from the notes---also in \cite{GreveBlatter2009,JouvetRappaz2011}---but now allowing any Glen exponent $n\ge 1$:
\begin{align}
\nabla \cdot \mathbf{u} &= 0 &&\text{\emph{incompressibility}} \label{incompressible} \\
- \nabla \cdot \tau + \nabla p &= \rho \mathbf{g} &&\text{\emph{stress balance}} \label{forcebalance} \\
D\bu &= A_n |\tau|^{n-1} \tau &&\text{\emph{Glen flow law}} \label{flowlaw}
\end{align}
Here $A_n$ is the $n$-dependent ice softness; in Table 1 we have $A = A_3 = 10^{-16} \,\text{Pa}^{-3}\,\text{a}^{-1} = 3.1689 \times 10^{-24} \,\text{Pa}^{-3}\,\text{s}^{-1}$ while generally the units of $A_n$ are $\text{Pa}^{-n}\,\text{s}^{-1}$.

Though notation generally follows Table 1 in the notes, some of the usage here is more general or flexible.  This especially regards how tensors and tensor norms are used.  Here are the concrete definitions and relationships wherein $\sigma$, $\tau$, $I$, and $D\bu$ all denote tensors and $|D\bu|$, $|\tau|$ denote tensor norms:
\begin{align*}
\sigma &= \tau - p\,I \\
(D\bu)_{ij} &= \frac{1}{2} \left((u_i)_{x_j} + (u_j)_{x_i}\right) \\
|\tau|^2 &= \frac{1}{2} \tau^\top \tau = \frac{1}{2} \tau_{ij} \tau_{ij} \\
|D\bu|^2 &= \frac{1}{2} (D\bu)^\top D\bu = \frac{1}{2} (D\bu)_{ij} (D\bu)_{ij}
\end{align*}
In the last two expressions the Einstein summation is used, either over indices $i,j=1,2,3$ or $i,j=1,2$ according to the context.

Also recall the viscosity form of flow law which is equation (15) in the notes.  Here that equation is
\begin{equation}
\tau = 2\nu D\bu = B_n |D\bu|^{\frac{1}{n} - 1} D\bu  \label{viscflowlaw}
\end{equation}
where $B_n = A_n^{-1/n}$ is the ice hardness.  With this expression we rewrite equations \eqref{forcebalance}, \eqref{flowlaw} and eliminate the deviatoric stress tensor $\tau$:
\begin{equation}
- \nabla \cdot \left(B_n |D\bu|^{\frac{1}{n} - 1} D\bu\right) + \nabla p = \rho \mathbf{g} \label{stokes}
\end{equation}

FIXME strong form boundary conditions

The Stokes model in strong form is, from now on, equations \eqref{incompressible} and \eqref{stokes} with boundary conditions FIXME.  It applies on a domain $\Omega\subset \RR^3$ or $\Omega \subset \RR^2$ according to the context.  This domain must have a smooth enough boundary to apply the prescribed normal stress boundary condition but it is otherwise general.  In the model the solution is the pair $(\bu,p)$ where $\bu\in V$ and $p \in Q$ for function spaces $V,Q$ precisely identified in \cite{JouvetRappaz2011}.  In fact the weak formulation of the model is proven in \cite{JouvetRappaz2011} to be well-posed under reasonable assumptions which will be satisfied in the cases we consider.

This weak formulation is derived by multiplying the equations by arbitrary test functions and then integrating by parts using the divergence theorem.  Specifically we define a nonlinear functional by multiplying \eqref{incompressible} by test function $q\in Q$ and \eqref{stokes} by test function $\bv\in V$ and combining these equations.  Thereby we get a nonlinear functional which must be zero at the solution $(\bu,p)$:
\begin{equation}
F(\bu,p;\bv,q) = \int_\Omega - \left(\nabla \cdot \left(B_n |D\bu|^{\frac{1}{n} - 1} D\bu\right)\right)\cdot \bv + \nabla p \cdot \bv - \rho \mathbf{g} \cdot \bv - \left(\nabla \cdot \bu\right) q \label{nonfuncone}
\end{equation}
The integration-by-parts step uses the product rule $\nabla \cdot(f\bX) = \grad f\cdot \bX + f \nabla \cdot \bX$ and the divergence theorem
    $$\int_\Omega \nabla \cdot \bX = \int_{\partial \Omega} \bX \cdot \hbn$$
Recalling $\tau = B_n |D\bu|^{\frac{1}{n} - 1} D\bu$ and $\bv = (v_1,v_2,v_3)$ then
\begin{align*}
\int_\Omega \left(\nabla \cdot \tau\right)\cdot \bv &= \sum_{j=1}^3 \int_\Omega \nabla \cdot (\tau_{\circ j})\, v_j = \sum_{j=1}^3 \int_\Omega \nabla \cdot (\tau_{\circ j} v_j) - \tau_{\circ j} \nabla v_j \\
  &= \sum_{j=1}^3 \int_{\partial \Omega} (\tau_{\circ j} v_j) \cdot \hbn - \int_\Omega \tau_{\circ j} \nabla v_j = \int_{\partial \Omega} (\tau \hbn)\cdot \bv - \int_\Omega \tau \cdot \nabla \bv,
\end{align*}
where $\circ$ denotes an index iterated-over, and similarly
    $$\int_\Omega \nabla p \cdot \bv = \int_\Omega \nabla\cdot (p\,\bv) - p (\nabla \cdot \bv) = \int_{\partial \Omega} (p I\hbn)\cdot \bv - \int_\Omega p (\nabla \cdot \bv)$$
These expressions allow us to rewrite \eqref{nonfuncone} with an integral of the normal stress over the boundary:
\begin{align}
F(\bu,p;\bv,q) &= -\int_{\partial\Omega} ((\tau-pI) \hbn)\cdot \bv + \int_\Omega \tau \cdot \nabla \bv - p (\nabla \cdot \bv) - \rho \mathbf{g} \cdot \bv - \left(\nabla \cdot \bu\right) q \label{nonfunctwo}
\end{align}





\footnotesize

\bigskip
%from: \bibliographystyle{siam}

\begin{thebibliography}{6}

\bibitem{BaliseRaymond1985}
{\sc M.~Balise and C.~Raymond}, {\em Transfer of basal sliding variations to
  the surface of a linearly-viscous glacier}, J. Glaciol., 31 (1985),
  pp.~308--318.

\bibitem{Brownetal2013}
{\sc J.~Brown, B.~Smith, and A.~Ahmadia}, {\em Achieving textbook multigrid
  efficiency for hydrostatic ice sheet flow}, SIAM J. Sci. Computing,
  35 (2013), pp.~B359--B375.

\bibitem{Elmanetal2014}
{\sc H.~C. Elman and D.~J. Silvester and A.~J. Wathen}, {\em Finite Elements
  and Fast Iterative Solvers: with Applications in Incompressible Fluid Dynamics},
  Oxford University Press, 2nd~ed., 2014.

\bibitem{GreveBlatter2009}
{\sc R.~Greve and H.~Blatter}, {\em Dynamics of {I}ce {S}heets and {G}laciers},
  Advances in Geophysical and Environmental Mechanics and Mathematics,
  Springer, 2009.

\bibitem{JouvetRappaz2011}
{\sc G.~Jouvet and J.~Rappaz}, {\em Analysis and finite element approximation
  of a nonlinear stationary {S}tokes problem arising in glaciology}, Advances
  in Numerical Analysis, (2011).

\bibitem{Lengetal2012}
{\sc W.~Leng, L.~Ju, M.~Gunzburger, S.~Price, and T.~Ringler}, {\em A parallel
  high-order accurate finite element nonlinear {S}tokes ice sheet model and
  benchmark experiments}, J. Geophys. Res., 117 (2012).

\end{thebibliography}


\end{document}
