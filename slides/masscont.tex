
\section{mass continuity}

\begin{frame}{the most-basic shallow assumption}

\begin{columns}

\begin{column}{0.6\textwidth}
\begin{itemize}
\item there are many shallow theories: SIA, SSA, hybrids, Blatter, \dots
\item \emph{all} make one assumption not required in Stokes:

\begin{center}
\alert{the surface and base of the ice are given by functions}
\end{center}
    \begin{itemize}
    \item[$\circ$] namely $z=h(t,x,y)$ and $z=b(t,x,y)$
    \item[$\circ$] surface overhang is not allowed
    \item[$\circ$] most Stokes models make this assumption too
    \end{itemize}
\end{itemize}
\end{column}

\begin{column}{0.4\textwidth}
\includegraphics[width=1.0\textwidth]{sshape}

\scriptsize
\begin{center}
\emph{not shallow!}
\end{center}
\vspace{6mm}

\includegraphics[width=1.0\textwidth]{serac}

\begin{center}
\emph{not shallow! not a fluid!}
\end{center}
\end{column}
\end{columns}
\end{frame}


\begin{frame}{three equations for geometry change}

\begin{itemize}
\item let $a$ be the climatic (surface) mass balance function
    \begin{itemize}
    \item[$\circ$] $a>0$ is accumulation and $a<0$ is ablation
    \end{itemize}
\item let $s$ be the basal melt rate (mass balance) function
    \begin{itemize}
    \item[$\circ$] $s>0$ is basal melting and $s<0$ is freeze-on
    \end{itemize}
\item let $M=a-s$
    \begin{itemize}
    \item[$\circ$] ``climatic-basal mass balance function'' in glossary
    \end{itemize}
\item define the map-plane flux of ice,
	$$\bq = \int_{b}^{h} (u,v)\,dz = \overline{\mathbf{u}}\,H$$
\item the three equations for glacier geometry change:
\begin{align*}
&\text{surface kinematical} && h_t = a - u\big|_h h_x - v\big|_h h_y + w\big|_h  \\
&\text{base kinematical} && b_t = s - u\big|_b b_x - v\big|_b b_y + w\big|_b  \\
&\text{mass continuity} && H_t = M - \Div \bq
\end{align*}
\end{itemize}
\end{frame}


\begin{frame}{kinematic and mass continuity equations}

\begin{itemize}
\item[Q:] what does the ``most-basic shallow assumption'' get you?
\item[A:] any two equations imply the third:
\small
\begin{align*}
&\text{surface kinematical} && h_t = a - u\big|_h h_x - v\big|_h h_y + w\big|_h  \\
&\text{base kinematical} && b_t = s - u\big|_b b_x - v\big|_b b_y + w\big|_b  \\
&\text{mass continuity} && H_t = M - \Div \bq
\end{align*}
\normalsize

\bigskip
\item to show the equivalences:
  \begin{itemize}
  \item[$\circ$]  recall the incompressibility of ice
    $$u_x + v_y + w_z = 0$$
  \item[$\circ$]  use the Leibniz rule for differentiating integrals
  {\scriptsize
    $$\frac{d}{dx}\left(\int_{g(x)}^{f(x)} h(x,y)\,dy\right) = f'(x) h(x,f(x)) - g'(x) h(x,g(x)) + \int_{g(x)}^{f(x)} h_x(x,y)\,dy$$}
  \item[$\circ$]  it's an exercise
  \end{itemize}
\end{itemize}
\end{frame}


\begin{frame}{the mass continuity equation: a summary}

\begin{itemize}
\item most ice sheet models use the mass continuity equation and the base kinematical equation

\bigskip
\item regarding how to think about the \emph{mass continuity equation}
  $$H_t = M - \nabla \cdot (\overline{\mathbf{u}} H),$$
a summary:
  \begin{itemize}
  \item[$\circ$] its character depends on the stress balance
  \item[$\circ$] it \emph{is} a transport equation
  \item[$\circ$] \dots but it is a diffusion for frozen bed, large scale flows (SIA)
      \begin{itemize}
      \item if your fancy Stokes model is not diffusive in this case \dots \emph{it's wrong}
      \end{itemize}
  \item[$\circ$] it is not very diffusive for membrane stresses and low basal resistance (e.g.~SSA for ice shelves)
  \item[$\circ$] additional mass continuity equations are needed for liquid water on surface and base \dots with much shorter time scales \dots which gets complicated
  \end{itemize}

\medskip
\item there is \emph{not} much helpful theory on the transport problems in glaciology \dots maybe you will help find this theory!
\end{itemize}
\end{frame}


\begin{frame}{standard recipe for ice sheet models}

\begin{enumerate}
  \item use stress balance to compute velocity
      \begin{itemize}
      \item[$\circ$] often: get $(u,v)$, then compute $w$ from incompressibility
      \end{itemize}
  \item somewhere in here do other processes: thermodynamics, basal melt, calving, \dots
  \item decide on time-step $\Delta t$ from diffusivity $D$ \hfill (or: \emph{fixed} $\Delta t$, sadly)
  \item from velocity, surface balance, and base balance do time-step of mass continuity equation to get $H_t$
  \item update surface elevation: $h \gets h+H_t \Delta t$
  \item repeat at 1.
\end{enumerate}

\bigskip
\begin{itemize}
\item this paradigm is \alert{explicit time stepping of the whole model}
  \begin{itemize}
  \item[$\circ$]  it will always be low-performance
  \end{itemize}
\end{itemize}
\end{frame}
